\chapter{Pushdown Automata}
\setlength{\parindent}{0pt}
\setlength{\parskip}{10pt}
Everything here is very similiar to Hopcroft and Ullman.
\begin{definition}
  \label{def:PDA}
  \lean{PDA}
  A Pushdown Automaton (PDA) is a Tuple $(Q,\Sigma,\Gamma,\delta,q_0,Z_0,F)$ where
  \begin{enumerate}
    \item $Q$ is the finite set of \emph{states}
    \item $\Sigma$ is the alphabet of the input
    \item $\Gamma$ is the alphabet of the \emph{stack}
    \item $\delta : Q\times(\Sigma\cup\{\varepsilon\})\times\Gamma\to\mathcal{P}(Q\times\Gamma^*)$
      is the \emph{transition function}, fullfilling $|\delta(q,a,Z)|<\infty$ forall $q\in Q$,
      $a\in\Sigma$ and $Z \in \Gamma$
    \item $q_0\in Q$ is the \emph{initial state}
    \item $Z_0 \in \Gamma$ is the \emph{start symbol}
    \item $F\subseteq Q$ are the \emph{final states}
  \end{enumerate}
\end{definition}

\begin{definition}
  \label{def:PDA_conf}
  \lean{PDA.conf}
  We call a Tuple $(q,x,\alpha)\in Q\times\Sigma^*\times\Gamma^*$ a \emph{configuration} of the PDA
  M=$(Q,\Sigma,\Gamma,\delta,q_0,Z_0,F)$.


\end{definition}
\begin{definition}
  \label{def:PDA_reaches}
  \lean{PDA.reaches}
  We say $(q,x,\alpha)\vdash^1(p,y,\beta)$ or configuration $(q,x,\alpha)$ \emph{reaches} $(p,y,\beta)$
  \emph{in one step} iff there exist $a\in\Sigma\cup\{e\}$, $Z\in\Gamma$
  and $\nu,\mu \in \Gamma^*$ so that  $x=ay$, $\alpha=Z\nu$,
  $\beta=\mu\nu$ and $(p,\mu)\in\delta(q,a,Z)$

  For $n\in\mathbb{N}$ fullfilling $n\ge 2$
  we say $(q,x,\alpha)\vdash^n(p,y,\beta)$ or configuration $(q,x,\alpha)$ \emph{reaches} $(p,y,\beta)$
  \emph{in $n$ steps} iff there exist $n-1$ configurations $c_i$ so that
  $(q,x,\alpha)\vdash^1c_1\vdash^1\cdots\vdash^1c_{n-1}\vdash^1(p,y,\beta)$
  Additionally we say $(q,x,\alpha)\vdash^0(p,y,\beta)$ iff $(q,x,\alpha)=(p,y,\beta)$.

  Finally we say $(q,x,\alpha)\vdash(p,y,\beta)$ or configuration $(q,x,\alpha)$ \emph{reaches}
  $(p,y,\beta)$ iff there exists $n\in\mathbb{N}$ so that $(q,x,\alpha)\vdash^n(p,y,\beta)$.

\end{definition}

\begin{definition}
  \label{def:PDA_language}
  \lean{PDA.acceptsByEmptyStack, PDA.acceptsByFinalState}
  For a PDA $M$ $(Q,\Sigma,\Gamma,\delta,q_0,Z_0,F)$ we define $$L(M)=\{w\in\Sigma^*\mid
  \exists q\in F,\alpha\in\Gamma^*: (q_0,w,Z_0)\vdash(q,\varepsilon,\alpha)\}$$ the
  \emph {Language of the PDA accepted by final state} and $$N(M)=\{w\in\Sigma^*\mid
  \exists q\in Q: (q_0,w,Z_0)\vdash(q,\varepsilon,\varepsilon)\}$$ the
  \emph {Language of the PDA accepted by empty stack}.

\end{definition}
\begin{theorem}\label{thm:PDA_reaches}
  \lean{PDA.reaches_refl,PDA.reaches_trans}
  The relation $\vdash$ is \emph{reflexiv} and \emph{transitive}.
\end{theorem}
\begin{proof}
  To proof reflexivity use $0$,
  for transitivity have  $(q,x,\alpha)\vdash(p,y,\beta)$ and
  $(p,y,\beta)\vdash(r,z,\gamma)$. This means there exist $n,m\in\mathbb{N}$ so that
  $(q,x,\alpha)\vdash^n(p,y,\beta)$ and $(p,y,\beta)\vdash^m(r,z,\gamma)$.
  The case $n=0$ or $m=0$ is trivial. Otherwise there exist $n-1$ (possibly $0$) configurations
  $c_i$ and $m-1$ configurations $\tilde{c}_i$ as in definition \ref{def:PDA_reaches}.
  Use
  $$
  h_i =
  \begin{cases}
    c_i          & \text{if } i\le n-1 \\
    (p,y,\beta) & \text{if } i = n \\
    \tilde{c}_i  & \text{if } n+1\le i \le n+m-1
  \end{cases}
  $$
  as the $n+m-1$ required configurations.
\end{proof}
\begin{theorem}\label{thm:PDA_reaches_input_decreasing}
  \lean{PDA.decreasing_input}
  If $(q,x,\alpha)\vdash(p,y,\beta)$ then $\exists w\in\Sigma^*: x=wy$.
\end{theorem}
\begin{proof}
  Induction
\end{proof}
\begin{theorem}\label{thm:PDA_reaches_unconsumed_input}
  \lean{PDA.unconsumed_input}
  For every $w\in\Sigma^*$ we have $(q,x,\alpha)\vdash(p,y,\beta)$ iff $(q,xw,\alpha)\vdash(p,yw,\beta)$
\end{theorem}
\begin{proof}
  Have $c_i = (q_i,x_i,\alpha_i)$ as in definition \ref{def:PDA_reaches} and use
  $\tilde{c}_i = (q_i,x_iw,\alpha_i)$
\end{proof}
\begin{theorem}\label{thm:PDA_reaches_unconsumed_stack}
  %\lean{PDA.unconsumed_stack}
  For every $\gamma\in\Gamma^*$ we have $(q,x,\alpha)\vdash(p,y,\beta)$ iff
  $(q,x,\underbrace{\alpha}_{\ne\varepsilon} \gamma)\vdash^1c_1\cdots c_{n-1}\vdash^1(p,y,\beta\gamma)$ where
  $c_i = (q_i,x_i,\underbrace{\alpha_i}_{\ne \varepsilon} \gamma)$. This means
  $(q,x,\alpha\gamma)\vdash(p,y,\beta\gamma)$ without "looking" at $\gamma$.
\end{theorem}
\begin{proof}
  Have $c_i = (q_i,x_i,\alpha_i)$ as in definition \ref{def:PDA_reaches} and use
  $\tilde{c}_i = (q_i,x_i,\alpha_i\gamma)$
\end{proof}
\begin{theorem}\label{thm:PDA_empty_stack_of_final_state}
  %\lean{PDA.empty_stack_of_final_state}
  Let $M(Q,\Sigma,\Gamma,\delta,q_0,Z_0,F)$ be a PDA with $L=L(M)$ then there exists a PDA $M'$ so that $N(M')=L.$
\end{theorem}
\begin{proof}
  Let $M$ be a PDA, with language $L=L(M)$ we want to specify a PDA $M'$ so that
  $N(M')=L$. We define
  \[
  M' =(Q\cup\{q_0',q_f\},\Sigma,\Gamma\cup\{Z_0'\},\delta',q_0',Z_0',\{\})
  \]
  where
  \[
  \delta'(q,a,Z) =
  \begin{cases}
    \{(q_0,Z_0Z_0')\} & \text{if } q=q_0'\land a=\varepsilon \land Z=Z_0' \\
    \{(q_f,\varepsilon)\}\cup\delta(q,a,Z)  & \text{if } q\in F\land a=\varepsilon \\
    \{(q_f,\varepsilon)\} & \text{if } q=q_f \land a=\varepsilon \\
    \delta(q,a,Z) & \text{otherwise}
  \end{cases}
  \]
  We show now that $N(M')=L$, for the first inclusion let $w\in L(M)$ arbitrarily.
  This means by definition
  \ref{def:PDA_language} $(q_0,w,Z_0)\vdash_M(q,\varepsilon,\alpha)$ for some $q\in F$ and
  $\alpha\in\Gamma^*$. Recall from \ref{def:PDA_reaches} that this requires the existence
  of zero or more configurations $c_i=(q_i,x_i,\alpha_i)\in Q\times\Sigma^*\times\Gamma^*$
  fullfilling
  $(q_0,w,Z_0)\vdash^1_M c_1\vdash^1_M\cdots\vdash^1_M c_{n-1}\vdash^1_M(q,\varepsilon,\alpha)$.
  As $\delta\subseteq\delta'$ each of this moves is also valid in $\vdash^1_{M'}$.
  So we have $(q_0,w,Z_0)\vdash_{M'}(q,\varepsilon,\alpha)$, as $q\in F$ it is clear that
  $\{(q_f,\varepsilon)\}\in \delta(q,\varepsilon,A)$ where $A$ ist the first symbol of $\alpha$.
  So we conclude  $(q_0,w,Z_0)\vdash_{M'}(q_f,\varepsilon,\alpha)$.
  By applying theorem \ref{thm:PDA_reaches_unconsumed_stack} we now know
  $(q_0,w,Z_0Z_0')\vdash_{M'}(q_f,\varepsilon,\alpha Z_0')$
  and by $\forall A\in\Gamma^*:\{(q_f,\varepsilon)\}\in\delta(q_f,\varepsilon,A)$ we obtain
  $(q_0,w,Z_0Z_0')\vdash_{M'}(q_f,\varepsilon,\varepsilon)$. Finally we have
  $(q_0,Z_0)\in\delta(q_0',\varepsilon,Z_0')$ and this concludes the first direction of
  the proof. Note that the stack of $M'$ can not be empty during this computation, as only the rules
  for $q_f$ pop symbol $Z_0'$.


  For the other direction let  $w\in N(M')$ arbitrarily. This means again by definition
  \ref{def:PDA_language} $(q_0',w,Z_0')\vdash_M(q,\varepsilon,\varepsilon)$ for some $q\in Q$.
  So again we have $c_i=(q_i,x_i,\alpha_i)\in Q\times\Sigma^*\times\Gamma^*$ with
  $(q_0',w,Z_0')\vdash^1_{M'} c_1\vdash^1_{M'}\cdots\vdash^1_{M'} c_{n-1}\vdash^1_{M'}(q,\varepsilon,\alpha)$.

  We need some facts about the computation of $M'$. Namely the first step is always
  $(q_0',w,Z_0')\vdash^1_{M'} (q_0,w,Z_0Z_0')$ so we know that $c_1=(q_0,w,Z_0Z_0')$.
  Similarily the last step is alway
  $(q_f',\varepsilon,Z_0')\vdash^1_{M'} (q_f,\varepsilon,\varepsilon)$
  so again we know that $c_{n-1}=(q_f,\varepsilon,Z_0')$. Now with induction we can
  show that there exists $m\in\mathbb{N}$ and $\alpha\in\Gamma^*$ so that $m\le n-1$ and
  $c_{m}=(q,\varepsilon,\alpha Z_0')\vdash(q_f,\varepsilon,Z_0')$ and $q\neq q_f$.
  This means $c_{m}$ is the last step of the computation before empting the stack.
  By applying theorem \ref{thm:PDA_reaches_unconsumed_stack} we receive
  $(q_0,w,Z_0)\vdash_{M'}(q,\varepsilon,\alpha)$. Now this computation is
  also valid in $M$, as none of the states $q_0'$ or $q_f$ can be reached inbetween.
  This proves $w\in\L(M)$.

\end{proof}

\begin{theorem}\label{thm:PDA_of_CFG}
  %\lean{PDA.PDA_of_CFG}
  Let $G$ be a CFG with $L=L(G)$ then there exists a PDA $M$ so that $N(M)=L.$
\end{theorem}
\begin{proof}
  Let $G=(N,T,P,S)$ be a context free grammar. We construct a PDA $M$, and show
  N(M)=L(G). So $M=(Q,\Sigma,\Gamma,\delta,q_0,Z_0,F)$ is defined as follows:
  \[
    Q=\{q_0\} \qquad \Sigma = T \qquad \Gamma = T \cup N \qquad Z_0=S  \qquad F=\emptyset
  \]
  \[
    \delta(q_0,a,Z) =  \begin{cases}
      \{(q_0,\beta)\mid  Z\to\beta\in P\} & \text{if} \quad a=\varepsilon \land Z\in N \\
      \{(q_0,\varepsilon)\} & \text{if} \quad a\in T \land Z\in T \land a=Z \\
    \end{cases}
  \]
  We show now L(G)=N(M). So let $w\in L(G)$ be arbitrary. So we know there exists a sequence
  of leftmost derivations
  \[
    S \Rightarrow_G^1 \alpha_1 \Rightarrow_G^1 \cdots \Rightarrow_G^1 \alpha_n \Rightarrow_G^1 w
  \]
  by induction on the number of steps we show that there exists a computation
  \[
    (q_0,w,S) \vdash_M^1 c_1 \vdash_M^1 \cdots \vdash_M^1 c_m \vdash_M^1
    (q_0,\varepsilon,\varepsilon).
  \]
  We need however a slightly stronger induction hypothesis, instead of $S$ we will work
  with $\alpha \in (N\cup T)^*$ and $w \in T^*$.
  For the base case we have $\alpha \Rightarrow_G^1 w$ this means in $\alpha$ is exactly one
  nonterminal, so $\alpha=w_1Aw_2$, $w=w_1vw_2$ and  $A\to v\in P$.
  Per construction of $M$ we know $(q_0,v)\in\delta(q_0,\varepsilon,A)$ so
  $(q_0,vw_2,Aw_2)\vdash_M^1(q_0,vw_2,vw_2)$. By repeatedly applying $(q_0,a)\in\delta(q_0,a,a)$
  we also have $(q_0,w_1vw_2,w_1Aw_2)\vdash_M^1(q_0,vw_2,Aw_2)$ and
  $(q_0,vw_2,vw_2)\vdash_M^1(q_0,\varepsilon,\varepsilon)$. By transitivity and because of
  $\alpha = w_1Aw_2$ aswell as $w=w_1Aw_2$ we conclude the base case.
  Now assuming  $\alpha \Rightarrow_G^n w \quad\Longrightarrow \quad(q_0,w,\alpha)\vdash_M
  (q_0,\varepsilon,\varepsilon)$, we want to show the same for  $\alpha \Rightarrow_G^{n+1} w$.
  If $\alpha \Rightarrow_G^{n+1} w$ we know there exists a $\alpha_1 \in  (N\cup T)^*$ so that
  \[
    \alpha \Rightarrow_G^1 \alpha_1 \Rightarrow_G^n w.
  \]
  Because $\alpha \Rightarrow_G^1 \alpha_1$
  we can write $\alpha = w_1A\alpha'$ and  $\alpha_1 = w_1\beta\alpha_1'$ where $w_1\in T^*$,
  $\alpha',\alpha_1',\beta\in (N\cup T)^*$ and $A\to\beta \in P.$  This of course implies
  $w=w_1w'$ for some $w'\in T^*$.
  Similarily as in the base case
  we can now show $(q_0,w_1w',w_1A\alpha')\vdash_M(q_0,w',\beta\alpha')$. Now it suffices to show
  $(q_0,w',\beta\alpha')\vdash_M(q_0,\varepsilon,\varepsilon)$. As  $\alpha_1 \Rightarrow_G^n w$
  and $\alpha_1 = w_1\beta\alpha_1'$, $w=w_1w'$ we see that $\beta\alpha_1'\Rightarrow_G^m w'$ for some
  $m\le n$. By applying the induction hypothesis we have that $(q_0,w',\beta\alpha_1')\vdash_M(q_0,\varepsilon,\varepsilon)$.
  This concludes the induction and with $\alpha=S$ the first half of the proof.

  For the other direction let again $w\in T^*$ be arbitrary, we again show by induction
  on the number of computation steps $(q_0,w,\alpha)\vdash_M^n(q_0,\varepsilon,\varepsilon)$
  implies $\alpha \Rightarrow_G w$ for every $w\in T^*, \alpha \in (T\cup N)^*$.
  For the base case we have $(q_0,w,\alpha)\vdash_M^1(q_0,\varepsilon,\varepsilon)$. There
  are only to possiblities for this computation,
  either $w=\varepsilon$ and $\alpha = A\in N$ with $A\to\varepsilon\in P$ or $w=a\in T$
  and $\alpha=a$. Anyway we see $\alpha \Rightarrow_G w$. For the induction step
  we assume $(q_0,w,\alpha)\vdash_M^n(q_0,\varepsilon,\varepsilon)\quad\Longrightarrow \quad
  \alpha \Rightarrow_G w$ and $(q_0,w,\alpha)\vdash_M^{n+1}(q_0,\varepsilon,\varepsilon)$.
  So
  \[
    (q_0,w,\alpha)\vdash_M^1(q_0,w',\alpha_1)\vdash_M^n(q_0,\varepsilon,\varepsilon).
  \]
  Obviously there exists $w_1\in T^*$ so that $w=w_1w'$. By the induction hypothesis
  we have $\alpha_1 \Rightarrow_G w'$. We distinguish  two possibly cases for the first
  computation step: Either there are $A\in N$ and $\beta \in (T\cup N)^*$ so that
  $w=w'$ (that is $w_1=\varepsilon$), $\alpha=A\alpha'$,
  $\alpha_1=\beta\alpha'$ and $A\to\beta\in P$ or $w=aw'$, $\alpha=a\alpha_1$. In the first
  case we have $\alpha \Rightarrow_G \alpha_1$ and as already established
  $\alpha_1 \Rightarrow_G w'=w$. So $\alpha \Rightarrow_G w$. In the second case
  we have $\alpha = a \alpha' \Rightarrow_G a w' = w.$ So in either case we have the desired result.
  By applying this to $S\Rightarrow_G w$ we have $L(G)\subseteq N(M)$.


\end{proof}
\begin{theorem}\label{thm:CFG_of_PDA}
  %\lean{PDA.CFG_of_PDA}
  Let $M$ be a PDA with $L=N(M)$ then there exists a CFG $G$ so that $L(G)=L.$
\end{theorem}
\begin{proof}
  As in Hopcroft, Ullman
\end{proof}
