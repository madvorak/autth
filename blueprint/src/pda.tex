\chapter{Pushdown Automata}
\setlength{\parindent}{0pt}
\setlength{\parskip}{10pt}
Everything here is very similiar to Hopcroft and Ullman.
\begin{definition}
  \label{def:PDA}
  \lean{PDA}
  A Pushdown Automaton (PDA) is a Tuple $(Q,\Sigma,\Gamma,\delta,q_0,Z_0,F)$ where
  \begin{enumerate}
    \item $Q$ is the finite set of \emph{states}
    \item $\Sigma$ is the alphabet of the input
    \item $\Gamma$ is the alphabet of the \emph{stack}
    \item $\delta : Q\times(\Sigma\cup\{\varepsilon\})\times\Gamma\to\mathcal{P}(Q\times\Gamma^*)$
      is the \emph{transition function}, fullfilling $|\delta(q,a,Z)|<\infty$ forall $q\in Q$,
      $a\in\Sigma$ and $Z \in \Gamma$
    \item $q_0\in Q$ is the \emph{initial state}
    \item $Z_0 \in \Gamma$ is the \emph{start symbol}
    \item $F\subseteq Q$ are the \emph{final states}
  \end{enumerate}
\end{definition}

\begin{definition}
  \label{def:PDA_conf}
  \lean{PDA.conf}
  We call a Tuple $(q,x,\alpha)\in Q\times\Sigma^*\times\Gamma^*$ a \emph{configuration} of the PDA
  M=$(Q,\Sigma,\Gamma,\delta,q_0,Z_0,F)$.


\end{definition}
\begin{definition}
  \label{def:PDA_reaches}
  \lean{PDA.reaches}
  We say $(q,x,\alpha)\vdash^1(p,y,\beta)$ or configuration $(q,x,\alpha)$ \emph{reaches} $(p,y,\beta)$
  \emph{in one step} iff there exist $a\in\Sigma\cup\{e\}$, $Z\in\Gamma$
  and $\nu,\mu \in \Gamma^*$ so that  $x=ay$, $\alpha=Z\nu$,
  $\beta=\mu\nu$ and $(p,\mu)\in\delta(q,a,Z)$

  For $n\in\mathbb{N}$ fullfilling $n\ge 2$
  we say $(q,x,\alpha)\vdash^n(p,y,\beta)$ or configuration $(q,x,\alpha)$ \emph{reaches} $(p,y,\beta)$
  \emph{in $n$ steps} iff there exist $n-1$ configurations $c_i$ so that
  $(q,x,\alpha)\vdash^1c_1\vdash^1\cdots\vdash^1c_{n-1}\vdash^1(p,y,\beta)$
  Additionally we say $(q,x,\alpha)\vdash^0(p,y,\beta)$ iff $(q,x,\alpha)=(p,y,\beta)$.

  Finally we say $(q,x,\alpha)\vdash(p,y,\beta)$ or configuration $(q,x,\alpha)$ \emph{reaches}
  $(p,y,\beta)$ iff there exists $n\in\mathbb{N}$ so that $(q,x,\alpha)\vdash^n(p,y,\beta)$.

\end{definition}

\begin{definition}
  \label{def:PDA_language}
  \lean{PDA.acceptsByEmptyStack}

  For a PDA $M$ $(Q,\Sigma,\Gamma,\delta,q_0,Z_0,F)$ we define $$L(M)=\{w\in\Sigma^*\mid
  \exists q\in F,\alpha\in\Gamma^*: (q_0,w,Z_0)\vdash(q,\varepsilon,\alpha)\}$$ the
  \emph {Language of the PDA accepted by final state} and $$N(M)=\{w\in\Sigma^*\mid
  \exists q\in Q: (q_0,w,Z_0)\vdash(q,\varepsilon,\varepsilon)\}$$ the
  \emph {Language of the PDA accepted by empty stack}.

\end{definition}
\begin{theorem}\label{thm:PDA_reaches}
  %\lean{PDA.trans}
  The relation $\vdash$ is \emph{reflexiv} and \emph{transitive}.
\end{theorem}
\begin{proof}
  To proof reflexivity use $0$,
  for transitivity have  $(q,x,\alpha)\vdash(p,y,\beta)$ and
  $(p,y,\beta)\vdash(r,z,\gamma)$. This means there exist $n,m\in\mathbb{N}$ so that
  $(q,x,\alpha)\vdash^n(p,y,\beta)$ and $(p,y,\beta)\vdash^m(r,z,\gamma)$.
  The case $n=0$ or $m=0$ is trivial. Otherwise there exist $n-1$ (possibly $0$) configurations
  $c_i$ and $m-1$ configurations $\tilde{c}_i$ as in definition \ref{def:PDA_reaches}.
  Use
  $$
  h_i =
  \begin{cases}
    c_i          & \text{if } i\le n-1 \\
    (p,y,\beta) & \text{if } i = n \\
    \tilde{c}_i  & \text{if } n+1\le i \le n+m-1
  \end{cases}
  $$
  as the $n+m-1$ required configurations.
\end{proof}
\begin{theorem}\label{thm:PDA_reaches_input_decreasing}
  %\lean{PDA.input_decreasing}
  If $(q,x,\alpha)\vdash(p,y,\beta)$ then $\exists w\in\Sigma^*: x=wy$.
\end{theorem}
\begin{proof}
  Induction
\end{proof}
\begin{theorem}\label{thm:PDA_reaches_unconsumed_input}
  %\lean{PDA.unconsumed_input}
  If $(q,x,\alpha)\vdash(p,y,\beta)$ then $(q,xw,\alpha)\vdash(p,yw,\beta)$ for
  every $w\in\Sigma^*$
\end{theorem}
\begin{proof}
  Have $c_i = (q_i,x_i,\alpha_i)$ as in definition \ref{def:PDA_reaches} and use
  $\tilde{c}_i = (q_i,x_iw,\alpha_i)$
\end{proof}
\begin{theorem}\label{thm:PDA_reaches_unconsumed_stack}
  %\lean{PDA.unconsumed_stack}
  If $(q,x,\alpha)\vdash(p,y,\beta)$ then $(q,x,\alpha\gamma)\vdash(p,y,\beta\gamma)$ for
  every $\gamma\in\Gamma^*$
\end{theorem}
\begin{proof}
  Have $c_i = (q_i,x_i,\alpha_i)$ as in definition \ref{def:PDA_reaches} and use
  $\tilde{c}_i = (q_i,x_i,\alpha_i\gamma)$
\end{proof}
\begin{theorem}\label{thm:PDA_empty_stack_of_final_state}
  %\lean{PDA.empty_stack_of_final_state}
  Let $M(Q,\Sigma,\Gamma,\delta,q_0,Z_0,F)$ be a PDA with $L=L(M)$ then there exists a PDA $M'$ so that $N(M')=L.$
\end{theorem}
\begin{proof}
  Let $M$ be a PDA, with language $L=L(M)$ we want to specify a PDA $M'$ so that
  $N(M')=L$. We define
  \[
  M' =(Q\cup\{q_0',q_f\},\Sigma,\Gamma\cup\{Z_0'\},\delta',q_0',Z_0',\{\})
  \]
  where
  \[
  \delta'(q,a,Z) =
  \begin{cases}
    \{(q_0,Z_0Z_0')\} & \text{if } q=q_0'\land a=\varepsilon \land Z=Z_0' \\
    \{(q_f,\varepsilon)\}\cup\delta(q,a,Z)  & \text{if } q\in F\land a=\varepsilon \\
    \{(q_f,\varepsilon)\} & \text{if } q=q_f \land a=\varepsilon \\
    \delta(q,a,Z) & \text{otherwise}
  \end{cases}
  \]
  We show now that $N(M')=L$, for the first inclusion let $w\in L(M)$ arbitrarily.
  This means by definition
  \ref{def:PDA_language} $(q_0,w,Z_0)\vdash_M(q,\varepsilon,\alpha)$ for some $q\in F$ and
  $\alpha\in\Gamma^*$. Recall from \ref{def:PDA_reaches} that this requires the existence
  of zero or more configurations $c_i=(q_i,x_i,\alpha_i)\in Q\times\Sigma^*\times\Gamma^*$
  fullfilling
  $(q_0,w,Z_0)\vdash^1_M c_1\vdash^1_M\cdots\vdash^1_M c_{n-1}\vdash^1_M(q,\varepsilon,\alpha)$.
  As $\delta\subseteq\delta'$ each of this moves is also valid in $\vdash^1_{M'}$. TODO
  So we have $(q_0,w,Z_0)\vdash_{M'}(q,\varepsilon,\alpha)$, as $q\in F$ it is clear that
  $\{(q_f,\varepsilon)\}\in \delta(q,\varepsilon,A)$ where $A$ ist the first symbol of $\alpha$.
  So we conclude  $(q_0,w,Z_0)\vdash_{M'}(q_f,\varepsilon,\alpha)$.
  By applying theorem \ref{thm:PDA_reaches_unconsumed_stack} we now know
  $(q_0,w,Z_0Z_0')\vdash_{M'}(q_f,\varepsilon,\alpha Z_0')$
  and by $\forall A\in\Gamma^*:\{(q_f,\varepsilon)\}\in\delta(q_f,\varepsilon,A)$ we obtain
  $(q_0,w,Z_0Z_0')\vdash_{M'}(q_f,\varepsilon,\varepsilon)$. Finally we have
  $(q_0,Z_0)\in\delta(q_0',\varepsilon,Z_0')$ and this concludes the first direction of
  the proof. Note that $\alpha$ can not be empty during this computation, as only the rules
  for $q_f$ pop symbol $Z_0'$.

  


\end{proof}

\begin{theorem}\label{thm:PDA_of_CFG}
  %\lean{PDA.PDA_of_CFG}
  Let $G$ be a CFG with $L=L(G)$ then there exists a PDA $M$ so that $L(M)=L.$
\end{theorem}
\begin{theorem}\label{thm:CFG_of_PDA}
  %\lean{PDA.CFG_of_PDA}
  Let $M$ be a PDA with $L=N(M)$ then there exists a CFG $G$ so that $L(G)=L.$
\end{theorem}
