% In this file you should put the actual content of the blueprint.
% It will be used both by the web and the print version.
% It should *not* include the \begin{document}
%
% If you want to split the blueprint content into several files then
% the current file can be a simple sequence of \input. Otherwise It
% can start with a \section or \chapter for instance.

\chapter{Regular expressions}

\begin{theorem}\label{thm:my_first_theorem}
\lean{my_first_theorem}
\leanok
$aab \in a^* b^*$
\end{theorem}
\begin{proof}
\leanok
easy
\end{proof}

% \begin{theorem}\label{thm:regular_iff_regexp}
% %\lean{}
% %\leanok
% %\uses{}
% A language is regular iff it has a regular expression.
% \end{theorem}
% \begin{proof}
% %\leanok
% \uses{lem:regular_impl_regexp,lem:regexp_impl_regular}
% The text of my proof...
% \end{proof}

% \begin{lemma}\label{lem:regular_impl_regexp}
% %\lean{}
% %\leanok
% %\uses{}
% Every regular language has a regular expression.
% \end{lemma}
% \begin{proof}
% %\leanok
% %\uses{}
% \end{proof}

% \begin{lemma}\label{lem:regexp_impl_regular}
% %\lean{}
% %\leanok
% %\uses{}
% Every language that has a regular expression is regular.
% \end{lemma}
% \begin{proof}
% %\leanok
% %\uses{}
% \end{proof}

\chapter{Context-free languages}

\begin{definition}\label{def:mygrammar}
  \lean{mygrammar}
  \leanok
Define the context-free grammar $G_1$ as $S \to \varepsilon | a S b$.
\end{definition}

\begin{theorem}\label{thm:my_second_theorem}
  \lean{my_second_theorem}
  \leanok
  $\varepsilon \in L(G_1)$
\end{theorem}
\begin{proof}
  \leanok
  easy
\end{proof}

\begin{theorem}\label{thm:my_third_theorem}
  \lean{my_third_theorem}
  \leanok
  $ab \in L(G_1)$
\end{theorem}
\begin{proof}
  \leanok
  easy
\end{proof}
